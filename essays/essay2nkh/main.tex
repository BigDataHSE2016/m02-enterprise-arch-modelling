\documentclass[7pt]{article}
\usepackage[utf8]{inputenc}

\title{Porter's Five Forces Model analysis}
\author{Natalia Khapaeva}

\usepackage{natbib}
\usepackage{graphicx}

\begin{document}

\maketitle
\footnotesize


The Porter's five forces framework if often used for qualitative assessment of a company.
The model works on the industry level("business line"\citep{article1}).
The most common criticisms of the model are that the model considers the market to be perfect; the micro- and macro-economics factors are not taken up; the level of stochasticity is low.
But, in my opinion, the main problem with the model is about it's age. It was developed more than 30 years ago and many things had changed since that time.


First, the industries have significantly changed. We live in a global world now. The supply chains are global, the competitors may subsist in the different edges of the world and it makes the analysis much more complicated. The customers are also global and now the companies can compete with each other for the loyalty of the whole world.

Second, the speed of the technical progress has increased. The new industries appear, evolve and disappear faster than ever before, the new opportunities and market requests and trends are born, mutate and sink away leaving the traditional industries with minor or fundamental changes.
The 3D-printing industry can be mentioned as an example. It changes many industries - from building (the world's first 3D-printed building was opened in Dubai this summer) to medicine (the prosthetic devices are already being 3D-printed by Russian company "Can-touch"). So the traditional industries companies should react to this new trend or to start losing their positions.

Third, we live in the information age, and it changes the ways of collaborating and producing the products and the products themselves. IT is not only a tool anymore, it has become the main driver in many industries\citep{article1}.
The access to the information is much easier now than 30 years ago. Thus, the new business models appear and change fast and the competitors landscape in the market changes rapidly.
The ease of communications leads to the deregularization and blurring of the markets. A 3D-printing company can produce almost anything and can switch from one market focus to another following the trends in social media as it's switching costs are extremely low. Thus, the "Can-touch" company had switched from producing the individual 3D-printing offers to the children medical prosthesis in a few months.


To conclude, the last 30 years imposed the new requirements to all the components of the Porter's five forces model. The model is still used, but it's limitations and disadvantages should be kept in mind.

\scriptsize
\nocite{*}
\bibliographystyle{plain}
\bibliography{references}
\end{document}

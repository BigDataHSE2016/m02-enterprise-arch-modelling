\documentclass[7pt]{article}
\usepackage[utf8]{inputenc}

\title{IT is still alive and does matter.}
\author{Natalia Khapaeva}

\usepackage{natbib}
\usepackage{graphicx}

\begin{document}

\maketitle
\footnotesize

The article "IT doesn't matter" was published in 2003 and claimed that IT couldn't be a competitive advantage anymore $^2$.\\
Since 2003 many things had happened, but the IT has survived and is still can act as an advantage. \\

\noindent
So, what are my arguments against Carr's ones?\\
\noindent
First, IT now is not just a set of technologies that were made for business support. IT has extended the reality we live in, and that leaded to the new spheres of life appearance that are completely virtual. MMORPG, social networks, the Darknet, some closed virtual communities do not exist in physical reality but they create values - both material and virtual and create opportunities for new
businesses.\\

\noindent
Second, I disagree with the design of the argument about the comparison between electricity, railroads and IT. The argument is based on the similarity of the charts that display the dependency of the volume of production in these spheres on time. But how can the number of computers as physical devices be mapped to the capacity of the IT industry? There is no correlation between the amount of devices and the raise and fall of IT or the IT itself. \\

%Third, Carr speaks that ubiquity, not scarcity, is the problem with IT.
\noindent
Third, Carr says there is no competitive advantage to be gained through IT as anyone can buy the same technology. But implementing the new technology and merging it with existing ones can be a challenge - buying something is just the beginning of the way. And there is another argument against Carr's one: it is about only existing technologies but not the innovations. IT has accelerated the speed of creating new products and values, new technologies are born fast and still can provide a competitive advantage. The number$^2$ of IT unicorns$^3$ can be considered as a proof of my statement. \\

\noindent
Fourth, the argument about risks IT might create$^5$. Yes, realization of new technology or even a minor update of an existing system (like Royal Bank of Scotland’s systems went offline for days$^6$) can cause problems but the technology is not the one that should be blamed. In most cases the cause of the problem is the human factor (like a bug in software, for example). Industries with critical assets related to IT had developed standards and standard procedures and special frameworks which mission is to reduce those risks and/or get them controlled.\\

\noindent
But there is a point in the article that I'm agree with. The "wild investing" in IT is unacceptable. Buying and implementing a new technology just because it is modern and fashionable can't automatically lead to success. The profits and risks of using a new technology should be properly assessed.\\

\noindent
In fact I think that the article is just a disruption that was made in order to raise the discussion on the future of IT and shouldn't be understood literally.


\scriptsize
\nocite{*}
\bibliographystyle{plain}
\bibliography{references}
\end{document}
